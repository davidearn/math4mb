%%%%%%%%%%%%%%%%%%%%%%%%%%%%%%%%%%%%%%%%%%%%%%%%%%%%%%%
%% QUESTIONS FOR MATH 4MB/6MB ASSIGNMENT 3.          %%
%% The question texts are used in several documents: %%
%% assignment, solutions, template,                  %%
%% hence it is better to load them from this file.   %%
%%%%%%%%%%%%%%%%%%%%%%%%%%%%%%%%%%%%%%%%%%%%%%%%%%%%%%%

\newcommand{\SIRintro}{%
  Consider the standard SIR model with vital dynamics,
  \begin{subequations}\label{E:SIR}
    \begin{align}
      \frac{dS}{dt} &= \mu N -\frac{\beta}{N} SI - \mu S\\
      \noalign{\vspace{8pt}}
      \frac{dI}{dt} &= \frac{\beta}{N} SI - \gamma I - \mu I\\
      \noalign{\vspace{8pt}}
      \frac{dR}{dt} &= \gamma I - \mu R
    \end{align}
  \end{subequations}
  where $S$, $I$ and $R$ denote the numbers of susceptible, infectious
  and removed individuals, respectively, and $N=S+I+R$ is the total
  population size. The \emph{per capita} rates of birth and death are
  the same (both are equal to $\mu$). As usual, $\beta$ is the
  transmission rate and $\gamma$ is the recovery rate.
}

\newcommand{\SIRa}{%
 Prove that the population size $N$ is constant and that
  equations \eqref{E:SIR} are biologically well-defined, \ie the set
  $\Delta$ of biologically meaningful states is forward-invariant.
  (Note that you will need to begin by defining precisely the set
  $\Delta$ and the term ``forward-invariant''.)
  %% if $0\le S(0),I(0),R(0)\le N$ then $0\le S(t),I(t),R(t)\le N$ for
  %% all $t>0$.
}

\newcommand{\SIRb}{%
 Show that equations \eqref{E:SIR} are equivalent dynamically to
  equations for the proportions (rather than numbers) of individuals
  in each disease state. For the remainder of this problem, use the
  equations in proportional form.
}

\newcommand{\SIRc}{%
 Re-express the equations for proportions in \emph{dimensionless}
  form using the dimensionless time coordinate
  \begin{subequations}\label{E:dimparms}
    \begin{equation}
      \tau=(\gamma+\mu)t\,,
    \end{equation}
    and the dimensionless parameters
    \begin{align}
      \R_0 &= \frac{\beta}{\gamma+\mu} \,,\\
      \noalign{\vspace{5pt}}
      \eps &= \frac {\mu}{\gamma+\mu} \,.
    \end{align}
  \end{subequations}
  What are the biological meanings of $\tau$, $\R_0$ and $\eps$? Why are
  they good choices for non-dimensionalizing the equations? For a few
  diseases that you are familiar with, what is the order of magnitude of
  $\eps$?
}

\newcommand{\SIRd}{%
 Show that there are exactly two equilibria: the disease free
  equilibrium (DFE) at $(S,I)=(1,0)$ and an endemic equilibrium (EE)
  at $(S,I)=(\Shat,\Ihat)$, where $\Shat$ and $\Ihat$ can be expressed
  compactly in terms of $\R_0$ and $\eps$.  Are both equilibria always
  biologically relevant?
}

\newcommand{\SIRe}{%
 Show that the DFE is locally asymptotically stable (LAS) if
  $\R_0<1$ and the EE is LAS if $\R_0>1$.
}

\newcommand{\SIRf}{%
 Prove that the DFE is, in fact, globally asymptotically stable (GAS)
  if $\R_0\le1$. \emph{\underline{Hint}:} This requires some careful
  analysis. Begin by using the function $L(S,I)=I$, and Theorem~2 stated
  in Assignment 1 under ``Notes on Lyapunov Functions'', to prove that
  all initial states in $\Delta$ are attracted to the $S$ axis.
  \label{DFEGASprob}
}

\newcommand{\SIRg}{%
 Prove that the EE is GAS if $\R_0>1$. \quad\emph{\underline{Hint}:}
  Consider
  % 
  \begin{equation}\label{E:LyapFun}
    L(S,I) = S - \Shat\log{S} + I - \Ihat\log{I} \,,
  \end{equation}
  % 
  and convince yourself that condition (a) in Theorem~1 stated in
  Assignment 1 under ``Notes on Lyapunov Functions'' can be replaced with
  % 
  \begin{equation}\label{E:}
    L(X)>L(X_*) \quad \text{for all} \quad X\in\openset\setminus\{X_*\} \,.
  \end{equation}
  % 
  \emph{\underline{Note}:} By GAS we mean here that \emph{almost all}
  initial states are attracted to the EE. One way of making this precise
  is to say that the \emph{basin of attraction} of the EE is an open,
  dense subset of $\Delta$. You should completely describe the basins of
  attraction of both the EE and the DFE. Do your results make biological
  sense?
  \label{EEGASprob}
}

\newcommand{\SIRh}{%
 Prove that the approach to the EE occurs via damped oscillations
  if and only if $\eps<\eps^*$, where
  \begin{equation}\label{E:Gstar}
    \eps^* = \frac {4(\R_0-1)}{\R_0^2} \,.
  \end{equation}
  For which diseases that you are familiar with would you expect damped
  oscillations versus monotonic convergence to the equilibrium?
}

\newcommand{\SIRi}{%
 For $\eps<\eps^*$, derive expressions for the period of damped
  oscillations onto the EE and the $e$-folding time for decay of the
  amplitude of oscillation.  Use \Rlogo to make a plot that
  displays your results graphically for some biologically relevant and
  illustrative parameter values.
}

\long\def\SIRj{
  Prove that as $\R_0$ is increased from $0$ to $\infty$, three
  ``bifurcations'' occur. In addition, use \Rlogo to make a
  four-panel plot that illustrates the different dynamics (phase
  portraits) in each of the four $\R_0$ intervals that have distinct
  dynamics. (\emph{\underline{Hint}:} I suggest you choose $\eps=8/9$
  for this figure, but you should explain why this is a good choice.)
  
  \smallskip
  
  \emph{\underline{Theoretical note}:} The word ``bifurcation'' is in
  quotes above because many dynamicists would consider only one of the
  three transitions to be a genuine bifurcation (it happens to be a
  \emph{transcritical bifurcation}).  The other two dynamical
  transitions yield biologically relevant qualitative changes, but the
  phase portraits on either side of the ``bifurcation point'' are
  actually topologically conjugate.

  \smallskip
  
  \emph{\underline{\Rlogo note}:}
  Computing phase portraits should be easy based on code you've
  written for solving ODEs in \Rlogo ``from scratch''. However, if you
  wish, you can use the \texttt{phaseR} package (or another \Rlogo
  package of your choice) to make the phase portraits.
}

\newcommand{\SIRk}{%
  Are there real diseases that display recurrent epidemics for which
  the standard SIR model that you have studied in this problem might be
  adequate to explain the observed epidemic dynamics?  If so, which
  diseases?  If not, why not?
}

%%\bibliographystyle{vancouver}
%%\bibliography{4mba3_2019}
